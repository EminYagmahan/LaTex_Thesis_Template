%------------------------------------------------------------------------------------------------------------------
% Licencse Header
%------------------------------------------------------------------------------------------------------------------

% MIT License
% 
% Copyright (c) 2019 Emin Yagmahan
% 
% Permission is hereby granted, free of charge, to any person obtaining a copy
% of this software and associated documentation files (the "Software"), to deal
% in the Software without restriction, including without limitation the rights
% to use, copy, modify, merge, publish, distribute, sublicense, and/or sell
% copies of the Software, and to permit persons to whom the Software is
% furnished to do so, subject to the following conditions:
% 
% THE SOFTWARE IS PROVIDED "AS IS", WITHOUT WARRANTY OF ANY KIND, EXPRESS OR
% IMPLIED, INCLUDING BUT NOT LIMITED TO THE WARRANTIES OF MERCHANTABILITY,
% FITNESS FOR A PARTICULAR PURPOSE AND NONINFRINGEMENT. IN NO EVENT SHALL THE
% AUTHORS OR COPYRIGHT HOLDERS BE LIABLE FOR ANY CLAIM, DAMAGES OR OTHER
% LIABILITY, WHETHER IN AN ACTION OF CONTRACT, TORT OR OTHERWISE, ARISING FROM,
% OUT OF OR IN CONNECTION WITH THE SOFTWARE OR THE USE OR OTHER DEALINGS IN THE
% SOFTWARE.

%------------------------------------------------------------------------------------------------------------------
%  Document Information
%------------------------------------------------------------------------------------------------------------------
% Author: Emin Yagmahan
% Last Mod. Date: 03.10.2019


%------------------------------------------------------------------------------------------------------------------
%  Document Definitions
%------------------------------------------------------------------------------------------------------------------

%  KOMA-Script Klasse mit einseitigem Druck, Titelblatt, Inhaltverzeichnis etc.
\documentclass[11pt, paper=a4, parskip=half, titlepage, toc=listof, toc=bib, toc=idx, numbers=noendperiod]{scrreprt}

% Geometry Paket zur Definition der Seitengrößen
\usepackage[left=30mm,right=25mm,top=25mm,bottom=25mm]{geometry}
\usepackage[english, ngerman]{babel}
\usepackage{lmodern}
\usepackage[T1]{fontenc}
\usepackage[utf8]{inputenc}
\usepackage{makeidx}
\usepackage{url}

% Graphicx Paket zum Handeln von Abbilungen. Alle Abbildungen sind unter dem Pfad "Abbildungen/" abzulegen
\usepackage[]{graphicx}
\graphicspath{Abbildungen/}

\usepackage{textcomp, fixmath}
\usepackage{bm}
\usepackage{listings}
\lstset{basicstyle=\ttfamily\footnotesize, frame=single, numbers=left, tabsize=3, backgroundcolor=\color{light-yellow}, extendedchars, keywordstyle=\color{red}, commentstyle=\color{blue}, language={[LaTex]Tex}}
\usepackage[usenames]{xcolor}

% WIP
\usepackage[backend=biber, sorting=none]{biblatex}

\addbibresource{Literatur/Literaturverzeichnis.bib}

\usepackage{csquotes}
\usepackage{microtype}
\usepackage[colorlinks=true, linkcolor=black, citecolor=black]{hyperref}
\usepackage{textpos}
\usepackage{setspace}
\usepackage{svg}
\usepackage{blindtext}

% Showframe Paket eignet sich zum Debuggen, indem Seitenränder angezeigt werden
%\usepackage{showframe}

% Paket zur Verwaltung von Abkürzungen. Der Parameter "printonlyused" wird lediglich die verwendeten Abkürzungen in das Abkürzungsverzeichnis integrieren
\usepackage[printonlyused]{acronym}
\usepackage{scrhack}

% Angaben zur Abbildungs- und Tabellenunterschriften
% Hinweis: Die Bild- oder Tabellenunterschriften befinden sich linksbündig relativ zur Abbildung, nicht zur Seite.
\usepackage{siunitx,booktabs,threeparttable,caption}
\captionsetup[table]{font=scriptsize,justification=raggedright,singlelinecheck=off}
\captionsetup[figure]{font=scriptsize,justification=raggedright,singlelinecheck=off}
\sisetup{separate-uncertainty=true}
\addtokomafont{captionlabel}{\bfseries}

% Entfernt den Texteinzug beim Erstellen neuer Kapitel
\RedeclareSectionCommand[beforeskip = 0pt, afterindent = false,]{chapter}

\addtokomafont{chapter}{\rmfamily\bfseries}
\addtokomafont{section}{\rmfamily\bfseries}
\addtokomafont{subsection}{\rmfamily\bfseries}
\addtokomafont{chapterentry}{\rmfamily\bfseries}


% Diese Custom-Kommando erstellt einheitliche Unterschrift und Datenfelder (siehe Eidesstattliche Erklärung)
\newcommand*{\SignatureAndDate}[1]{%
    \vspace{2cm}
    \par\noindent\makebox[2.5in]{\hrulefill} \hfill\makebox[2.0in]{\hrulefill}%
    \par\noindent\makebox[2.5in][l]{#1}      \hfill\makebox[2.0in][l]{Ort, Datum}%
}%


%------------------------------------------------------------------------------------------------------------------
%  Document Beginn
%------------------------------------------------------------------------------------------------------------------

\begin{document}

% Römische Seitennummerierung
\pagenumbering{roman}
% Definition der Schriftarten und -größen für das Titelblatt
\addtokomafont{subject}{\normalfont\Large\rmfamily}
\addtokomafont{author}{\rmfamily}
\addtokomafont{date}{\rmfamily}
\addtokomafont{publishers}{\rmfamily}
\addtokomafont{title}{\rmfamily}
\addtokomafont{subtitle}{\normalfont\Large\rmfamily}


%------------------------------------------------------------------------------------------------------------------
%  Title Page Configuration (modify this section only!)
%------------------------------------------------------------------------------------------------------------------
\newcommand{\DegreeThesisType}{Masterthesis}
\newcommand{\TitleOfWork}{Ethische Aspekte der künstlichen Intelligenz in der Medizin}
\newcommand{\DegreeType}{Master of Science}
\newcommand{\Faculty}{Fachbereich Gesundheit}
\newcommand{\Organisation}{Technischen Hochschule Musterstadt}
\newcommand{\AuthorName}{Emin Yagmahan}
\newcommand{\PublicDateMY}{Januar 2020}
\newcommand{\Referent}{Prof. Dr. Max Mustermann}
\newcommand{\Korreferent}{Dr. Max Mustermann}


% Description of Thesis Type
\subject{\vspace{-1cm}\DegreeThesisType}
\title{
	%Title of Work
	{\vspace{0.5cm}\TitleOfWork}
}

% Adding Subtitle
\subtitle{\vspace{2.5cm}zur Erlangung des akademischen Grades \\
	\DegreeType \\
	\vspace{1.5cm}
	vorgelegt dem \\
	\Faculty \\
	der \Organisation
	\vspace{1cm}
}

% Adding  Author und Pub. Date
\author{\AuthorName}
\date{\vspace{0.5cm}im \PublicDateMY}

% Adding Publisher informations
\publishers{
	\vspace{5cm}
	\begin{flushleft}
		\begin{tabular}{ll}
			Referent:    & \Referent \\
			Korreferent: & \Korreferent     
		\end{tabular}
	\end{flushleft}
}
\maketitle

% Setze Zeilenabstand auf 1.5x
\setstretch{1.5}

% Einbinden der obligatorischen Kapitel
\chapter*{Sperrvermerk}

%------------------------------------------------------------------------------------------------------------------
%  Sperrvermerk Konfiguration (modify this section only!)
%------------------------------------------------------------------------------------------------------------------
\newcommand{\CompanyName}{Mustermanns' GmbH}


%------------------------------------------------------------------------------------------------------------------
%  Beginn des Benutzertest
%------------------------------------------------------------------------------------------------------------------
Die vorliegende Masterarbeit mit dem Titel „\TitleOfWork“ beinhaltet interne und vertrauliche Informationen des Unternehmens \CompanyName. Eine Einsicht in diese \DegreeThesisType ist nicht gestattet. Ausgenommen davon sind die betreuenden Dozenten sowie die befugten Mitglieder des Prüfungsausschusses. Eine Veröffentlichung und Vervielfältigung der Masterarbeit - auch in Auszügen oder elektronisch - ist nicht gestattet. Ausnahmen von dieser Regelung bedürfen einer Genehmigung durch das Unternehmen \CompanyName.
\chapter*{Eidesstattliche Erklärung}

%------------------------------------------------------------------------------------------------------------------
%  Beginn des Benutzertextes
%------------------------------------------------------------------------------------------------------------------
Hiermit versichere ich, die vorliegende Arbeit selbstständig und unter ausschließlicher Verwendung der angegebenen Literatur und Hilfsmittel erstellt zu haben. Die Arbeit wurde bisher in gleicher oder ähnlicher Form keiner anderen Prüfungsbehörde vorgelegt und auch nicht veröffentlicht.

% Einheitliches Unterschriftsfeld hinzufügen
\SignatureAndDate{Unterschrift des Autors}
\chapter*{Abstract}

%------------------------------------------------------------------------------------------------------------------
%  Beginn des Benutzertextes
%------------------------------------------------------------------------------------------------------------------
\blindtext
\chapter*{Kurzfassung}

%------------------------------------------------------------------------------------------------------------------
%  Beginn des Benutzertextes
%------------------------------------------------------------------------------------------------------------------
\blindtext
% Fügt Inhaltsangabe hinzu. Der Titel wird automatisch ergänzt
\tableofcontents

% Beginn der arabischen Seitennummerierung
\pagenumbering{arabic}

% Einbinden der inhaltlichen Kapitel. 
\chapter{Einleitung}


%------------------------------------------------------------------------------------------------------------------
%  Beginn des Benutzertextes
%------------------------------------------------------------------------------------------------------------------
\blindtext

\section{Motivation}
\blindtext

\subsection{Mehr Motivation}
\blindtext
Für weitere Beispiele siehe Tabelle ~\ref{perflogcross}.


\begin{table}[h]
	\centering
	\begin{threeparttable}
		\begin{tabular}{llll}
			\toprule
			{Sensitivity} & {Specificity} & {BACC}& {Threshold} \\
			\midrule
			0.555 $\pm$ 0.118 & 0.924 $\pm$ 0.028 & 0.738 $\pm$ 0.059 & 0.235 $\pm$ 0.029 \\
			0.560 $\pm$ 0.110 & 0.927 $\pm$ 0.029 & 0.743 $\pm$ 0.054 & 0.234 $\pm$ 0.030 \\
			0.527 $\pm$ 0.126 & 0.924 $\pm$ 0.033 & 0.725 $\pm$ 0.068 & 0.231 $\pm$ 0.031 \\ 
			\bottomrule
		\end{tabular}
	\caption{Fiktive Tabelle gemäß den Anforderungen des Fachbereiches}
	\label{perflogcross}
	\end{threeparttable}
\end{table}

\blindtext

Referenziert wird die folgende Abbildung ~\ref{plot}.

\begin{figure}[h]

	\centering
     \captionbox{Fiktive Abbildung gemäß den Anforderungen des Fachbereiches\label{plot}}{

	\includesvg[width = 400pt]{Abbildungen/Plot_Beispiel.svg}
     }
\end{figure}

Das ist ein Beispiel wie man das Paper von Pan \& Tompins zur Echtzeit-QRS Detektion zitiert~\cite{pan_real-time_1985}.
\input{Kapitel/Zielsetzung.tex}
\chapter{Material und Methoden}

%------------------------------------------------------------------------------------------------------------------
%  Beginn des Benutzertextes
%------------------------------------------------------------------------------------------------------------------
\blindtext
\chapter{Ergebnisse}

%------------------------------------------------------------------------------------------------------------------
%  Beginn des Benutzertextes
%------------------------------------------------------------------------------------------------------------------
\blindtext

\chapter{Diskussion}

%------------------------------------------------------------------------------------------------------------------
%  Beginn des Benutzertextes
%------------------------------------------------------------------------------------------------------------------
\blindtext
\chapter{Zusammenfassung}

%------------------------------------------------------------------------------------------------------------------
%  Beginn des Benutzertextes
%------------------------------------------------------------------------------------------------------------------
\blindtext

% Huinzufügen der abschließenden obligatorischen Kapitel
% [WIP] Leider keine schönere Methode gefunden, welche die folgenden Kapitel nicht nummeriert, aber dennoch in das Inhaltverzeichnis integriert. Folgende Methode funktioniert.
\chapter*{Danksagung}

%------------------------------------------------------------------------------------------------------------------
%  Beginn des Benutzertextes
%------------------------------------------------------------------------------------------------------------------
\blindtext
\addcontentsline{toc}{chapter}{Danksagung}

% Fügt ein Abbildungsverzeichnis hinzu. Titel wird automatisch ergänzt.
\listoffigures
% Fügt ein Abbildungsverzeichnis hinzu. Titel wird automatisch ergänzt.
\listoftables

% Fügt ein Abbildungsverzeichnis hinzu. Titel wird automatisch ergänzt.
\chapter*{Abkürzungsverzeichnis}


% Nutze die längste Abkürzung als Option. Das garantiert eine schönere Splatung Zwischen Abkürzung und Ausschreibung
\begin{acronym}[PACS]
	\acro{KDE}{K Desktop Enviroment}
	\acro{SQL}{Structured Query Language}
	\acro{PACS}{Picture Archiving and Communication System}
\end{acronym}


Anbei befindet sich ein Beispiel, wie Abkürzungen in {\LaTeX} zu verwenden sind. Die in der Main.tex gesetzte Option \textit{printonlyused} manipuliert das Abkürzungsverzeichnis, sodass Abkürzungen, die zwar definiert, aber nicht im Text vorhanden sind, nicht hinzugefügt werden. Siehe folgendes Beispiel:

Die schönste Desktopumgebung ist \ac{KDE}. Ein medizinisches Informationssystem lautet \ac{PACS}. Die Datenbanksprache SQL ist nicht als Abkürzung definiert. Obwohl ein Akronym existiert, wird dies nicht dem Abkürzungsverzeichnis hinzugefügt. Abkürzungen werden bei der ersten Nutzung ausgeschrieben und stehen als Link zur Verfügung.
\addcontentsline{toc}{chapter}{Abkürzungsverzeichnis}

\input{Literatur/Literaturverzeichnis.tex}
\addcontentsline{toc}{chapter}{Literaturverzeichnis}

\input{Vorgabe/Anhang.tex}
\addcontentsline{toc}{chapter}{Anhang}

\end{document}

