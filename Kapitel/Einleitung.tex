\chapter{Einleitung}


%------------------------------------------------------------------------------------------------------------------
%  Beginn des Benutzertextes
%------------------------------------------------------------------------------------------------------------------
\blindtext

\section{Motivation}
\blindtext

\subsection{Mehr Motivation}
\blindtext
Für weitere Beispiele siehe Tabelle ~\ref{perflogcross}.


\begin{table}[h]
	\centering
	\begin{threeparttable}
		\begin{tabular}{llll}
			\toprule
			{Sensitivity} & {Specificity} & {BACC}& {Threshold} \\
			\midrule
			0.555 $\pm$ 0.118 & 0.924 $\pm$ 0.028 & 0.738 $\pm$ 0.059 & 0.235 $\pm$ 0.029 \\
			0.560 $\pm$ 0.110 & 0.927 $\pm$ 0.029 & 0.743 $\pm$ 0.054 & 0.234 $\pm$ 0.030 \\
			0.527 $\pm$ 0.126 & 0.924 $\pm$ 0.033 & 0.725 $\pm$ 0.068 & 0.231 $\pm$ 0.031 \\ 
			\bottomrule
		\end{tabular}
	\caption{Fiktive Tabelle gemäß den Anforderungen des Fachbereiches}
	\label{perflogcross}
	\end{threeparttable}
\end{table}

\blindtext

Referenziert wird die folgende Abbildung ~\ref{plot}.

\begin{figure}[h]

	\centering
     \captionbox{Fiktive Abbildung gemäß den Anforderungen des Fachbereiches\label{plot}}{

	\includesvg[width = 400pt]{Abbildungen/Plot_Beispiel.svg}
     }
\end{figure}

Das ist ein Beispiel wie man das Paper von Pan \& Tompins zur Echtzeit-QRS Detektion zitiert~\cite{pan_real-time_1985}.